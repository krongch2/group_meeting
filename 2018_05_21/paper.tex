\documentclass{beamer}
%
% Choose how your presentation looks.
%
% For more themes, color themes and font themes, see:
% http://deic.uab.es/~iblanes/beamer_gallery/index_by_theme.html
%

\mode<presentation>
{
  \usetheme{default} % or try Darmstadt, Madrid, Warsaw, ...
  \usecolortheme{default} % or try albatross, beaver, crane, ...
  \usefonttheme{serif} % or try default, structurebold, ...
  \setbeamertemplate{navigation symbols}{}
  \setbeamertemplate{caption}[numbered]
}
\newcommand*\vf[1]{\mathbf{#1}}
\usepackage[english]{babel}
\usepackage[utf8x]{inputenc}
\usepackage{amsmath,mathtools,esint,bm}
\usepackage{todonotes}
\usepackage{blindtext}
\usepackage[version=3]{mhchem}
\newcommand{\mick}[1]{\textcolor{red}{[ #1 ]}}

\AtBeginSection[]{
  \begin{frame}
  \vfill
  \centering
  \begin{beamercolorbox}[sep=8pt,center,shadow=true,rounded=true]{title}
    \usebeamerfont{title}\insertsectionhead\par%
  \end{beamercolorbox}
  \vfill
  \end{frame}
}

\title[]{Charge Density Waves in Cuprates}
\author{Mick Krongchon}
\institute{University of Illinois at Urbana-Champaign}
\date{\today}

\begin{document}

\begin{frame}
\titlepage
\end{frame}

\begin{frame}{Outline}
\begin{itemize}
% \item Motivation
% \item What is a phonon?
\item Why care about charge density waves?
\item Summary of Miao and Dean's paper
\item My research
\item Conclusion
\end{itemize}
\end{frame}

\begin{frame}{Why care about charge density waves (CDWs)?}
\begin{itemize}
\item CDW is a phase transition at low $T$
\item Metals undergo a phase transition when they are cooled
% \item i.e. Iron and nickel become FM. Lead and Al become superconductors
\item Since 1970s, many layered materials have been found to undergo this phase transition
\item Want to study CDW
\end{itemize}
\end{frame}

% \begin{frame}{Classical 1D chain with two masses}
% \begin{itemize}
% \item Each atom interacts only with its nearest neighbors
% \item The system equations are
% \begin{align*}
% M_1 \frac{d^2}{dt^2} u_1^l = K(u_2^l - u_1^l) - K(u_1^l - u_2^{l - 1}) = K(u_2^l - 2u_1^l + u_2^{l - 1}). \\
% M_2 \frac{d^2}{dt^2} u_2^l = K(u_1^{l + 1} - u_2^l) - K(u_2^l - u_1^l) = K(u_1^{l + 1} - 2u_2^l + u_1^l).
% \end{align*}
% \begin{figure}
% \includegraphics[width=3in]{figs/1d_diagram.png}
% \caption{\label{fig:1d_diagram}} % Alternating masses $M_1$ and $M_2$ interacting with nearest neighbors
% \end{figure}
% \end{itemize}
% \end{frame}

% \begin{frame}{Phonon branches}
% \begin{itemize}
% \item Plug in $u_j^l = A_j e^{i(kla - \omega t)}$
% \item Solve for $\omega$ that gives no trivial solutions (setting $M_2 = r M_1$):
% \begin{align}
% \omega = \sqrt{\frac{K}{M_1}}
% \sqrt{\frac{1 + r \pm \sqrt{1 + 2r\cos{ka} + r^2}}{r}}.
% \end{align}
% \begin{figure}
% \includegraphics[width=3in]{figs/1d_dispersion.pdf}
% \caption{\label{fig:1d_dispersion} $\omega(k)$ when $r = 3$}
% \end{figure}
% \end{itemize}
% \end{frame}

% \begin{frame}{Crystalline lattices are \textbf{not} 1D chains}
% \begin{itemize}
% % \item They are not 3D mass-spring lattices either
% % \item But the idea is correct given small deviations
% % \item Let $\vf{u}^1$ ... $\vf{u}^N$ describe the displacement of ions $1$ ... $N$ from their equilibrium locations $\vf{R}^1$ ... $\vf{R}^N$
% \item Take the energy functional to be
% \begin{align*}
% \mathcal{E}(\vf{u}^1, \vf{u}^2~...~\vf{u}^N) = \mathcal{E}(u_x^1, u_y^1, u_z^1, u_x^2, u_y^2, u_z^2~...~u_x^N, u_y^N, u_z^N).
% \end{align*}
% \item The second-order Taylor series expansion around $\vf{O} = (\vf{0}^1, \vf{0}^2~...~\vf{0}^N)$ is
% \begin{align}
% \mathcal{E} &= \mathcal{E}(\vf{O}) + \sum_{\alpha,i} \frac{\partial \mathcal{E}(\vf{O})}{\partial u_\alpha^i}u_\alpha^i + \frac{1}{2} \sum_{\alpha,i} \sum_{\beta,j} \frac{\partial^2 \mathcal{E}(\vf{O})}{\partial u_\alpha^i \partial u_\beta^j} u_\alpha^i u_\beta^j, \\
% &= \mathcal{E}_c + \frac{1}{2} \sum_{\alpha,i} \sum_{\beta,j} u_\alpha^i \Phi_{\alpha \beta}^{i j} u_\beta^{j}, \label{eq:3d_energy},
% \end{align}
% where $\alpha$ and $\beta$ range over $x$, $y$, and $z$, and $i$ and $j$ range from $1$ to $N$.
% \end{itemize}
% \end{frame}

% \begin{frame}{}
% \begin{itemize}
% \item Taking the derivative to find force
% \begin{align*}
% F_x &=&&-\frac{\partial \mathcal{E}}{\partial u_x^l} = -\sum_i
% \begin{bmatrix}
% \Phi_{xx}^{li} & \Phi_{xy}^{li} & \Phi_{xz}^{li}
% \end{bmatrix}
% \begin{bmatrix}
% u_x^i & u_y^i & u_z^i
% \end{bmatrix}^\text{T}.
% \end{align*}
% \item Similarly for $F_y$ and $F_z$, the equation of motion is therefore
% \begin{align}
% \boxed{M \ddot{\vf{u}}^l = -\sum_i
% \begin{bmatrix}
% \Phi_{xx}^{li} & \Phi_{xy}^{li} & \Phi_{xz}^{li} \\
% \Phi_{yx}^{li} & \Phi_{yy}^{li} & \Phi_{yz}^{li} \\
% \Phi_{zx}^{li} & \Phi_{zy}^{li} & \Phi_{zz}^{li}
% \end{bmatrix}
% \begin{bmatrix}
% u_x^i \\
% u_y^i \\
% u_z^i
% \end{bmatrix} = -\sum_i \Phi^{li} \vf{u}^i}, \label{eq:phonon_motion}
% \end{align}
% where the subscripts are suppressed in the $3 \times 3$ matrix $\Phi^{li}$.
% \item Plug in, $\vf{u}^l = \vf{A} e^{i (\vf{k} \cdot \vf{R}^l - \omega t)}$
% \begin{align}
% M \omega^2 \vf{A} &= \sum_{l^{\prime}} \Phi^{l l^{\prime}} e^{i \vf{k} \cdot (\vf{R}^{l^{\prime}} - \vf{R}^l)} \vf{A} = \Phi(\vf{k}) \vf{A}, \label{eq:phonon_motion_matrix}
% \end{align}
% \end{itemize}
% \end{frame}

% \begin{frame}{What is charge density waves (CDW)}
% \begin{itemize}
% \item The electron density in metals is uniform
% \item The equilibrium positions form a perfectly periodic lattice
% \item When $T < T_c$, the Fermi surface becomes unstable
% \item The instability may result in periodic charge density modulation called CDW
% \item In 1950s, Fr\"{o}hlich tried to explain SC that electrons and lattices move together
% % \item No unifying description for CDW in different systems?
% \end{itemize}
% \end{frame}

\begin{frame}{Peierls' picture: 1D chain}
\begin{itemize}
% \item In 1930s, Peierls described the instability in a 1D chain of equally spaced atoms
% \item The only zero energy transition is from $k_F$ to $-k_F$, % where $k_F$ is the Fermi wave number
\item The Fermi points are at $k_F = \pm \pi / 2 a$ % connected by a vector $q = 2 k_F$
% \item The disturbance with $q = 2 k_F$ changes spacing to $2 a$
\item Gap occurs at $k = \pm \pi / 2 a$ when $T < T_c$
\item Metal-to-insulator transition at $T_c$ is called \textbf{Peierls' transition}
\begin{figure}
\includegraphics[width=3in]{figs/density_fermi.pdf}
\caption{\label{fig:density_fermi} (a, c) $T > T_c$ (b, d) $T < T_c$}
\end{figure}
\end{itemize}
\end{frame}

\begin{frame}{Free electron gas model}
\begin{itemize}
\item \textbf{Lindhard response function} $\chi(q)$ describes free electron gas' response
\item 1D electron gas is unstable
\begin{figure}
\includegraphics[width=2.5in]{figs/lindhart.png}
\caption{\label{fig:lindhart} Real part of Linhard function for 1D, 2D and 3D free electron gas models}
\end{figure}
\end{itemize}
\end{frame}

\begin{frame}{Kohn anomaly}
\begin{itemize}
\item In 1959, Kohn: The excitations at $2 k_F$ will screen any lattice motion with this wave vector
\item The phonon modes near $2 k_F$ will drop to a lower energy
\item This is called \textbf{phonon softening}
%  This strong renormalization of the phonon due to interactions with an electron system is referred to as the Kohn anomaly
\begin{figure}
\includegraphics[width=2in]{figs/kohn_anomaly.png}
\caption{\label{fig:kohn_anomaly} Phonon energy of 1D chain at different $T$}
\end{figure}
\end{itemize}
\end{frame}

\begin{frame}{Miao and Dean's \textit{Incommensurate Phonon Anomaly and the Nature of Charge Density Waves in Cuprates} (2018)}
\begin{itemize}
\item \textit{Goal:} Find unifying description for superconducting cuprates
\item CDW instabilities are common superconducting cuprates
\item CDWs have different ordering wave vectors
\item They measured the $T$ dependence of the low-energy phonons in La$_{1.875}$Ba$_{0.125}$CuO$_4$ and compared it to YBa$_2$Cu$_3$O$_{6 + \delta}$
\end{itemize}
\end{frame}

\begin{frame}{Inelastic x-ray scattering measurement of $S(\vf{Q}, \omega)$}
\begin{itemize}
\item Why measured $S(\vf{Q}, \omega)$? ``Because they can" - Lucas
% \item Dashed lines are calculated using DFPT with PBE functional
\begin{figure}
\includegraphics[width=3.5in]{figs/exp_sq.pdf}
\caption{\label{fig:exp_sq} Phonon dynamic structure factor at $T = 300~\mathrm{K}$}
\end{figure}
\end{itemize}
\end{frame}

\begin{frame}{Example fits of spectra at $K$ = 0.09, 0.23, and 0.31 r.l.u. at different $T$}
\begin{figure}
\includegraphics[width=3.5in]{figs/exp_fit.pdf}
\caption{\label{fig:exp_fit} }
\end{figure}
\end{frame}

\begin{frame}{Peaks at different $T$}
\begin{figure}
\includegraphics[width=2.2in]{figs/exp_E_k.png}
\caption{\label{fig:exp_E_k} $Q_{\text{CDW}} = 0.23~\mathrm{r.l.u.}$}
\end{figure}
\end{frame}

\begin{frame}{$T$ dependence at $K = 0.23~\mathrm{r.l.u.}$}
\begin{itemize}
\item Since they found that $M2$ has a large component of $c$-axis displacements, they suggest that $c$-axis coupled most strongly to the CDW
\begin{figure}
\includegraphics[width=3.5in]{figs/exp_T_dependence.png}
\caption{\label{fig:exp_T_dependence} $T_{\text{CDW}} = 55~\mathrm{K}$}
\end{figure}
\end{itemize}
\end{frame}

\begin{frame}{Maximum phonon softening moves with increasing $T$}
\begin{itemize}
\item They conclude that $Q_{\text{CDW}} = 0.238~\mathrm{r.l.u.}$ at $T = 55~\mathrm{K}$, and $Q_{\text{CDW}} = 0.3~\mathrm{r.l.u.}$ at $T = 300~\mathrm{K}$
\item Which is similar to a wave vector reported in YBa$_2$Cu$_3$O$_{6 + \delta}$
\begin{figure}
\includegraphics[width=2.5in]{figs/exp_E_k_zoomed.pdf}
\caption{\label{fig:exp_E_k_zoomed} }
\end{figure}
\end{itemize}
\end{frame}

\begin{frame}{My research}
\begin{itemize}
\item \textit{Goal:} Identify phonon-softening modes of charge density waves in \ce{SrCuO2} and \ce{La2CuO4}???
\item Step 1: Optimize the geometry of undoped and 1/4-doped \ce{SrCuO2} using DFT with 2x2 unit cells and PBEsol functional
\item Step 2: Find phonon frequencies and eigenvectors at the gamma point
\end{itemize}
\end{frame}

\begin{frame}{Identify how modes change after doping}
\begin{itemize}
\item The matrix element of $i$th AFM mode and $j$th Flip mode
\begin{align}
e_{ij} = \sum_{k = 1}^{48} v_{\text{AFM}, ik} \times v_{\text{Flip}, jk}
\end{align}
\item The two modes whose dot product is more than 0.8 are consider to be the same
\begin{table}
% \centering
% \caption{\label{tab:dot} Dot products of  \ce{SrCuO2} (normallized within each column)}
% \caption{\label{tab:widgets}Example values of the magnetic susceptibility, $\chi_v$.}
\begin{tabular}{ccccccccccc}
AFM / Flip & 1 & 2 & 3 & 4 & 5 & 6 & 7 & 8 & 9 & 10 \\
% \colrule \\
1 & 0 & 0 & 0.895 & 0 & 0 & 0 & 0 & 0 & 0 & 0 \\
2 & 0.895 & 0 & 0 & 0 & 0 & 0 & 0 & 0 & 0 & 0 \\
3 & 0 & 0.875 & 0 & 0 & 0 & 0 & 0.001 & 0 & 0 & 0.047 \\
4 & 0 & 0 & 0 & 0.805 & 0 & 0 & 0 & 0 & 0 & 0 \\
5 & 0 & 0 & 0 & 0 & 0.806 & 0 & 0 & 0 & 0 & 0 \\
6 & 0 & 0 & 0 & 0 & 0 & 0 & 0 & 0 & 1.000 & 0 \\
\end{tabular}
\end{table}
\end{itemize}
\end{frame}

\begin{frame}{The change in phonon frequency}
\begin{figure}
\includegraphics[width=4.2in]{figs/sco_afm_flip_freq.pdf}
\caption{\label{fig:sco_afm_flip_freq} See AVI files}
\end{figure}
\end{frame}

\begin{frame}{Conclusion}
\begin{itemize}
\item Kohn: Charge density waves affect phonon frequencies
\item Dean: CDW in LCO and YBCO might be of the same origin (they both have $Q_{\text{CDW}} = 0.3~\mathrm{r.l.u.} at T = 300~\mathrm{K}$)
\item My research: Doping weakens the Sr-O and Cu-O bonds
\end{itemize}
\end{frame}

\end{document}
