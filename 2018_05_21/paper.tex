\documentclass{beamer}
%
% Choose how your presentation looks.
%
% For more themes, color themes and font themes, see:
% http://deic.uab.es/~iblanes/beamer_gallery/index_by_theme.html
%

\mode<presentation>
{
  \usetheme{default} % or try Darmstadt, Madrid, Warsaw, ...
  \usecolortheme{default} % or try albatross, beaver, crane, ...
  \usefonttheme{serif} % or try default, structurebold, ...
  \setbeamertemplate{navigation symbols}{}
  \setbeamertemplate{caption}[numbered]
}
\newcommand*\vf[1]{\mathbf{#1}}
\usepackage[english]{babel}
\usepackage[utf8x]{inputenc}
\usepackage{amsmath,mathtools,esint,bm}
\usepackage{todonotes}
\usepackage{blindtext}
\newcommand{\mick}[1]{\textcolor{red}{[ #1 ]}}

\AtBeginSection[]{
  \begin{frame}
  \vfill
  \centering
  \begin{beamercolorbox}[sep=8pt,center,shadow=true,rounded=true]{title}
    \usebeamerfont{title}\insertsectionhead\par%
  \end{beamercolorbox}
  \vfill
  \end{frame}
}

\title[]{Charge Density Waves in Cuprates}
\author{Mick Krongchon}
\institute{University of Illinois at Urbana-Champaign}
\date{\today}

\begin{document}

\begin{frame}
\titlepage
\end{frame}

\begin{frame}{Outline}
\begin{itemize}
\item What is a phonon?
\item What is a charge density wave?
\item Summary of Miao and Dean's paper
\item My research
\item Summary
\end{itemize}
\end{frame}

\begin{frame}{Classical 1D chain with two masses}
\begin{itemize}
\item Each atom interacts only with its nearest neighbors
\item The system equations are
\begin{align*}
M_1 \frac{d^2}{dt^2} u_1^l = K(u_2^l - u_1^l) - K(u_1^l - u_2^{l - 1}) = K(u_2^l - 2u_1^l + u_2^{l - 1}). \\
M_2 \frac{d^2}{dt^2} u_2^l = K(u_1^{l + 1} - u_2^l) - K(u_2^l - u_1^l) = K(u_1^{l + 1} - 2u_2^l + u_1^l).
\end{align*}
\begin{figure}
\includegraphics[width=3in]{figs/1d_diagram.png}
\caption{\label{fig:1d_diagram} Alternating masses $M_1$ and $M_2$ interacting with nearest neighbors}
\end{figure}
\end{itemize}
\end{frame}

\begin{frame}{Phonon branches}
\begin{itemize}
\item Assuming a solution of the form $u_j^l = A_j e^{i(kla - \omega t)}$,
we solve for $\omega$ that gives no trivial solutions (setting $M_2 = r M_1$):
\begin{align}
\omega = \sqrt{\frac{K}{M_1}}
\sqrt{\frac{1 + r \pm \sqrt{1 + 2r\cos{ka} + r^2}}{r}}.
\end{align}
\begin{figure}
\includegraphics[width=3in]{figs/1d_dispersion.pdf}
\caption{\label{fig:1d_dispersion} Vibrational frequencies of a chain with two alternating masses when $r = 3$}
\end{figure}
\end{itemize}
\end{frame}

\begin{frame}{Origin of charge density waves (CDW)}
\begin{itemize}
\item The electron density in metals is uniform
\item The equilibrium positions form a perfectly periodic lattice
\item When $T < T_c$, the Fermi surface becomes unstable
\item The instability may result in periodic charge density modulation called CDW
\item In 1950s, Fr\"{o}hlich tried to explain SC that electrons and lattices move together
\item No unifying description for CDW in different systems?
\end{itemize}
\end{frame}

\begin{frame}{Peierls' picture: 1D chain}
\begin{itemize}
% \item In 1930s, Peierls described the instability in a 1D chain of equally spaced atoms
% \item The only zero energy transition is from $k_F$ to $-k_F$, % where $k_F$ is the Fermi wave number
\item The Fermi points are at $k_F = \pm \pi / 2 a$ % connected by a vector $q = 2 k_F$
\item The disturbance with $q = 2 k_F$ changes spacing to $2 a$
\item Gap occurs at $k = \pm \pi / 2 a$
\item Metal-to-insulator transition at $T_c$ is called \textbf{Peierls' transition}
\begin{figure}
\includegraphics[width=3in]{figs/density_fermi.pdf}
\caption{\label{fig:density_fermi} (a, c) $T > T_c$ (b, d) $T < T_c$}
\end{figure}
\end{itemize}
\end{frame}

\begin{frame}{The free electron gas model}
\begin{itemize}
\item \textbf{Lindhart response function} $\chi(q)$ describes free electron gas' response
\item 1D electron gas is unstable
\begin{figure}
\includegraphics[width=2.5in]{figs/lindhart.png}
\caption{\label{fig:lindhart} Real part of Linhard function for 1D, 2D and 3D free electron gas models}
\end{figure}
\end{itemize}
\end{frame}

\begin{frame}{Kohn anomaly}
\begin{itemize}
\item In 1959, Kohn: The excitations at $2 k_F$ will screen any lattice motion with this wave vector
\item The phonon modes near $2 k_F$ will be renormalized to have lower energy
\item This is called \textbf{phonon softening}
%  This strong renormalization of the phonon due to interactions with an electron system is referred to as the Kohn anomaly
\begin{figure}
\includegraphics[width=2in]{figs/kohn_anomaly.png}
\caption{\label{fig:kohn_anomaly} Phonon energy of 1D chain at different $T$}
\end{figure}
\end{itemize}
\end{frame}

% \begin{frame}{Classical lattice vibration}
% \begin{itemize}
% % \item Solid elasticity theory fails when the deformation wavelength is comparable to inter-atomic distances
% % \item Phonons are traveling waves in crystals
% \item Consider a one-dimensional chain of ions of mass~$M$ sitting at equilibrium distance~$a$ connected by springs of constant~$K$
% \end{itemize}
% \end{frame}

% \begin{frame}
% \begin{itemize}
% \item Let $u^l$ denote the deviation of ion $l \in [0, ..., N - 1]$ from its equilibrium location, with periodic boundary conditions implying that
% \begin{align}
% u^N = u^0 \label{eq:periodic_cond}
% \end{align}
% The system equation is given by
% \begin{align}
% M \frac{d^2}{dt^2}u^l = K(u^{l + 1} - u^l) - K(u^l - u^{l - 1}) \label{eq:1d_chain}
% \end{align}
% The equation is solved by plane waves. Let
% \begin{align}
% u^l = A e^{i(k l a - \omega t)}. \label{eq:planewave}
% \end{align}
% Eq.~(\ref{eq:periodic_cond}) requires
% \begin{align}
% k N a = 2 \pi n,~k = \frac{2 \pi n}{N a},~n \in [0, ..., N - 1].
% \end{align}
% \end{itemize}
% \end{frame}

% \begin{frame}
% \begin{itemize}
% \item Substituting Eq.~(\ref{eq:planewave}) into Eq.~(\ref{eq:1d_chain}) gives
% \begin{align}
% -M \omega^2 A e^{i(k l a - \omega t)} &= [K(e^{i k a} - 1) - K(1 - e^{- i k a})] A e^{i (k l a - \omega t)} \notag \\
% -M \omega^2 &= 2 K [\cos(i k a) - 1] = 2 K \left[-2 \sin^2{\left(\frac{i k a}{2}\right)}\right] \notag \\
% \omega &= 2 \sqrt{\frac{K}{M}} \left|\sin{\left(\frac{k a}{2}\right)}\right|
% \end{align}
% \end{itemize}
% \end{frame}

% \begin{frame}{Two characteristic features of this solution}

% 1. $\omega \propto |k|$ when $k \rightarrow 0$.
% \begin{itemize}
% \item Linear elasticity for waves much longer than the interatomic spacing
% \item $d \omega / dk = a \sqrt{K / M} = c$ near the origin is the sound speed
% \end{itemize}
% 2. The solution repeats as a function of $k$ with period $2 \pi / a$.
% \begin{itemize}
% \item The phonon problem concerns waves moving the medium that is periodic with period $a$
% \item The same reason that energies of electrons are periodic functions of wave vector $\vf{k}$
% \end{itemize}
% \end{frame}


% \begin{frame}{}
% \begin{itemize}
% \item Assuming a solution of the form $u_j^l = A_j e^{i(kla - \omega t)}$, we get
% \begin{align}
% -\omega^2 M_1 A_1 &= K[A_2 - 2 A_1 + A_2 e^{- i k a}]. \\
% -\omega^2 M_2 A_2 &= K[A_1e^{i k a} - 2 A_2 + A_1].
% \end{align}
% Grouping the coefficients of $A_1$ and $A_2$ gives
% \begin{align}
% (\omega^2 M_1 - 2 K)A_1 + (K + K e^{- i k a})A_2 = 0. \label{eq:system_1} \\
% (K e^{i k a} + K)A_1 + (\omega^2 M_2 - 2 K)A_2 = 0. \label{eq:system_2}
% \end{align}
% Set the determinant of the system to zero:
% \begin{align*}
% M_1 M_2 \omega^4 - 2K (M_1 &+ M_2)\omega^2 + 4K^2 \\
% &- K^2(1 + e^{- i k a})(1 + e^{i k a}) = 0.
% \end{align*}
% Solve for $\omega$:
% \begin{align*}
% M_1 M_2 \omega^4 - 2K (M_1 + M_2)\omega^2 + K^2[2 - (e^{- i k a} + e^{i k a})] &= 0. \\
% M_1 M_2 \omega^4 - 2K (M_1 + M_2)\omega^2 + 2 K^2(1 - \cos{ka}) = 0.
% \end{align*}
% \end{itemize}
% \end{frame}

% \begin{frame}{}
% \begin{align}
% \omega^2 &= \frac{K(M_1 + M_2) \pm K\sqrt{(M_1 + M_2)^2 - 2 M_1 M_2 (1 - \cos{ka})}}{M_1 M_2}. \notag \\
% \omega &= \sqrt{K}\sqrt{\frac{M_1 + M_2 \pm \sqrt{M_1^2 + 2 M_1 M_2 \cos{ka} + M_2^2}}{M_1 M_2}}. \label{eq:disp_rel_basis}
% \end{align}
% Setting $M_2 = r M_1$ gives
% \begin{align}
% \omega = \sqrt{\frac{K}{M_1}}
% \sqrt{\frac{1 + r \pm \sqrt{1 + 2r\cos{ka} + r^2}}{r}} \label{eq:disp_rel_basis_m1}.
% \end{align}
% \begin{itemize}
% \item The two solutions of Eq.~(\ref{eq:disp_rel_basis}) are two \textit{branches} of the phonon dispersion relation
% \end{itemize}
% \end{frame}

% \begin{frame}{}
% \begin{itemize}
% \item The \textit{acoustic branch} vanishes linearly near $k = 0$ and corresponds to ordinary sound
% \item The \textit{optical branch} is restricted higher frequencies since in solids these phonons are characteristically excited by light
% \end{itemize}
% \end{frame}

% \begin{frame}{}
% \begin{itemize}
% \item For small $k$, the acoustic branch takes the form
% \begin{align}
% \omega_{\text{-}}(k) &= \sqrt{K}\sqrt{\frac{M_1 + M_2 - \sqrt{M_1^2 + 2 M_1 M_2 \left[1 - \frac{(k a)^2}{2}\right] + M_2^2}}{M_1 M_2}} \notag \\
% &= \sqrt{K}\sqrt{\frac{M_1 + M_2 - (M_1 + M_2)\sqrt{1 - M_1 M_2 \frac{(ka)^2}{(M_1 + M_2)^2}}}{M_1 M_2}} \notag \\
% &= \sqrt{K}\sqrt{\frac{M_1 + M_2 - (M_1 + M_2)\left[1 - \frac{M_1 M_2 (ka)^2}{2 (M_1 + M_2)^2}\right]}{M_1 M_2}} \notag \\
% &= \sqrt{K}\sqrt{\frac{\frac{M_1 M_2 (ka)^2}{2 (M_1 + M_2)}}{M_1 M_2}} = k a \sqrt{\frac{K}{2 (M_1 + M_2)}}.
% \end{align}
% Plugging this back into Eq.~(\ref{eq:system_2}) gives $A_2 = (1 + i k a / 2) A_1$.
% \end{itemize}
% \end{frame}

% \begin{frame}{}
% \begin{itemize}
% \item For small $k$, the optical branch takes the form
% \begin{align}
% \omega_{\text{+}}(k) &= \sqrt{K} \sqrt{\frac{M_1 + M_2 + (M_1 + M_2)}{M_1 M_2}} \notag \\
% &= \sqrt{\frac{2 K (M_1 + M_2)}{M_1 M_2}}.
% \end{align}
% Plugging this back into Eq.~(\ref{eq:system_1}) gives
% \begin{align}
% A_2 = -\frac{M_1}{M_2} \left(1 + \frac{i k a}{2}\right) A_1.
% \end{align}
% \item Thus, for the acoustic mode, atoms within the unit cell move essentially in unison, while for the optical mode, atoms within the unit cell vibrate out of phase.
% \end{itemize}
% \end{frame}

\begin{frame}{Crystalline lattices are \textbf{not} 1D chains}
\begin{itemize}
% \item They are not 3D mass-spring lattices either
% \item But the idea is correct given small deviations
\item Let $\vf{u}^1$ ... $\vf{u}^N$ describe the displacement of ions $1$ ... $N$ from their equilibrium locations $\vf{R}^1$ ... $\vf{R}^N$
\item Take the energy functional to be
\begin{align*}
\mathcal{E}(\vf{u}^1, \vf{u}^2~...~\vf{u}^N) = \mathcal{E}(u_x^1, u_y^1, u_z^1, u_x^2, u_y^2, u_z^2~...~u_x^N, u_y^N, u_z^N).
\end{align*}
\item The second-order Taylor series expansion around $\vf{O} = (\vf{0}^1, \vf{0}^2~...~\vf{0}^N)$ is
\begin{align*}
\mathcal{E} = \mathcal{E}(\vf{O}) + \sum_{\alpha,i} \frac{\partial \mathcal{E}(\vf{O})}{\partial u_\alpha^i}u_\alpha^i + \frac{1}{2} \sum_{\alpha,i} \sum_{\beta,j} \frac{\partial^2 \mathcal{E}(\vf{O})}{\partial u_\alpha^i \partial u_\beta^j} u_\alpha^i u_\beta^j,
\end{align*}
where $\alpha$ and $\beta$ range over $x$, $y$, and $z$, and $i$ and $j$ range from $1$ to $N$.
\end{itemize}
\end{frame}

\begin{frame}{}
\begin{itemize}
\item The first term is the cohesive energy, which is a constant denoted by $\mathcal{E}_c$
\item Since the lowest energy state is a minimum as a function of ion locations, $\partial \mathcal{E}(\vf{O}) / \partial u_\alpha^i$ is zero for every $\alpha$ and $i$. The second term is zero
\begin{align}
\mathcal{E} = \mathcal{E}_c + \frac{1}{2} \sum_{\alpha,i} \sum_{\beta,j} u_\alpha^i \Phi_{\alpha \beta}^{i j} u_\beta^{j}, \label{eq:3d_energy}
\end{align}
where $\Phi_{\alpha \beta}^{i j}$ is given by
\begin{align}
\Phi_{\alpha \beta}^{i j} = \frac{\partial^2 \mathcal{E}(\vf{O})}{\partial u_\alpha^i \partial u_\beta^j}.
\end{align}
\end{itemize}
\end{frame}

% \begin{frame}{Periodic boundary conditions simplify lattice vibration calculations}
% \begin{itemize}
% \item Along any of the three primitive vectors, every physical quantity is assumed to repeat
% \item All points in the equilibrium crystal are equivalent
% \item The crystal has no surfaces or boundary
% \item Consequence: $\Phi_{\alpha \beta}^{i j}$ depends only on $\vf{R}^i - \vf{R}^j$ \mick{why?}
% \item If one displaces just two ions $\vf{u}^i$ and $\vf{u}^j$, leaving all others in equilibrium locations, the resulting energy can depend only on their relative locations
% \end{itemize}
% \end{frame}

% \begin{frame}{The dynamical equation for phonons}
% \begin{itemize}
% \item From Eq.~(\ref{eq:3d_energy}),
% \begin{align*}
% \mathcal{E} = \mathcal{E}_c + \frac{1}{2} \sum_{i = 1}^N \sum_{j = 1}^N (
% &u_x^i \Phi_{x x}^{i j} u_x^j + u_x^i \Phi_{x y}^{i j} u_y^j + u_x^i \Phi_{x z}^{i j} u_z^j + \\
% &u_y^i \Phi_{y x}^{i j} u_x^j + u_y^i \Phi_{y y}^{i j} u_y^j + u_y^i \Phi_{y z}^{i j} u_z^j + \\
% &u_z^i \Phi_{z x}^{i j} u_x^j + u_z^i \Phi_{z y}^{i j} u_y^j + u_z^i \Phi_{z z}^{i j} u_z^j).
% \end{align*}
% \item The $x$ component of the force on ion $l$ is $-\partial \mathcal{E} / \partial u_x^l$
% \item We split the sum into three parts: 1) $j = j = l$, 2) $i \neq l$ and $j = l$, and 3) $i = l$ and $j \neq l$
% \end{itemize}
% \end{frame}

% \begin{frame}{}
% \begin{itemize}
% \item Dropping the terms without $u_x^l$ because they give zero contributions to the derivative, we get
% \begin{align*}
% \frac{\partial \mathcal{E}}{\partial u_x^l}
% &=&&\frac{1}{2} \frac{\partial}{\partial u_x^l} (u_x^l \Phi_{xx}^{ll}u_x^l + 2 u_x^l \Phi_{xy}^{ll} u_y^l + 2 u_x^l \Phi_{xz}^{ll} u_z^l) + \\
% & &&\frac{1}{2}\frac{\partial}{\partial u_x^l} \sum_{i \neq l} (u_x^i \Phi_{xx}^{il} u_x^l + u_y^i \Phi_{yx}^{il} u_x^l + u_z^i \Phi_{zx}^{il} u_x^l) + \\
% & &&\frac{1}{2}\frac{\partial}{\partial u_x^l} \sum_{j \neq l} (u_x^l \Phi_{xx}^{lj} u_x^j + u_x^l \Phi_{xy}^{lj} u_y^j + u_x^l \Phi_{xx}^{lj} u_x^j) \\
% &=&&\frac{1}{2}(2 \Phi_{xx}^{ll} u_x^l + 2 \Phi_{xy}^{ll} u_y^l + 2 \Phi_{xz}^{ll} u_z^l) + \\
% & &&\frac{1}{2}\sum_{i \neq l}(u_x^i \Phi_{xx}^{il} + u_y^i \Phi_{yx}^{il} + u_z^i \Phi_{zx}^{il}) + \\
% & &&\frac{1}{2}\sum_{j \neq l}(\Phi_{xx}^{lj} u_x^j + \Phi_{xy}^{lj} u_y^j + \Phi_{xz}^{lj} u_z^j).
% \end{align*}
% \end{itemize}
% \end{frame}

\begin{frame}{}
\begin{itemize}
\item Taking the derivative to find force
\begin{align*}
% \frac{\partial \mathcal{E}}{\partial u_x^l} &=&&(\Phi_{xx}^{ll} u_x^l + \Phi_{xy}^{ll} u_y^l + \Phi_{xz}^{ll} u_z^l) + \\
% & &&\sum_{i \neq l} (\Phi_{xx}^{il} u_x^i + \Phi_{yx}^{il} u_y^i + \Phi_{zx}^{il} u_x^i) \\
% &=&&\sum_{i} (\Phi_{xx}^{li} u_x^i + \Phi_{xy}^{li} u_y^i + \Phi_{xz}^{li} u_z^i). \\
F_x &=&&-\frac{\partial \mathcal{E}}{\partial u_x^l} = -\sum_i
\begin{bmatrix}
\Phi_{xx}^{li} & \Phi_{xy}^{li} & \Phi_{xz}^{li}
\end{bmatrix}
\begin{bmatrix}
u_x^i & u_y^i & u_z^i
\end{bmatrix}^\text{T}.
\end{align*}
\item Similarly for $F_y$ and $F_z$, the equation of motion is therefore
\begin{align}
\boxed{M \ddot{\vf{u}}^l = -\sum_i
\begin{bmatrix}
\Phi_{xx}^{li} & \Phi_{xy}^{li} & \Phi_{xz}^{li} \\
\Phi_{yx}^{li} & \Phi_{yy}^{li} & \Phi_{yz}^{li} \\
\Phi_{zx}^{li} & \Phi_{zy}^{li} & \Phi_{zz}^{li}
\end{bmatrix}
\begin{bmatrix}
u_x^i \\
u_y^i \\
u_z^i
\end{bmatrix} = -\sum_i \Phi^{li} \vf{u}^i}, \label{eq:phonon_motion}
\end{align}
where the subscripts are suppressed in the $3 \times 3$ matrix $\Phi^{li}$.
\end{itemize}
\end{frame}

\begin{frame}{}
\begin{itemize}
\item Eq.~(\ref{eq:phonon_motion}) is solved by plane waves in a 3D crystal, so we take
\begin{align}
\vf{u}^l = \vf{A} e^{i (\vf{k} \cdot \vf{R}^l - \omega t)}.
\end{align}
Substituting in Eq.~(\ref{eq:phonon_motion}) gives
\begin{align}
-M \omega^2 \vf{A} e^{i \vf{k} \cdot \vf{R}^l} e^{- i \omega t} &= -\sum_{l^{\prime}} \Phi^{l l^{\prime}} \vf{A} e^{i(\vf{k} \cdot \vf{R}^{l^{\prime}} - \omega t)} \\
M \omega^2 \vf{A} &= \sum_{l^{\prime}} \Phi^{l l^{\prime}} e^{i \vf{k} \cdot (\vf{R}^{l^{\prime}} - \vf{R}^l)} \vf{A} \notag \\
&= \Phi(\vf{k}) \vf{A}, \label{eq:phonon_motion_matrix}
\end{align}
where we replace the dummy index $i$ with $l^{\prime}$ to distinguish from the imaginary unit, and
\begin{align}
\Phi(\vf{k}) = \sum_{l^{\prime}} e^{i \vf{k} \cdot (\vf{R}^l - \vf{R}^{l^{\prime}})} \Phi^{l l^{\prime}}. \label{eq:dyn_matrix}
\end{align}
\item Eq.~(\ref{eq:phonon_motion_matrix}) is a matrix equation for the polarization vector $\vf{A}$
\end{itemize}
\end{frame}

% \begin{frame}{}
% \begin{itemize}
% \item The matrix $\Phi(\vf{k})$ is real and symmetric, and therefore has three orthogonal eigenvectors for every $k$
% \item Let $\vf{A}_{\vf{k} 1}$, $\vf{A}_{\vf{k} 2}$, and $\vf{A}_{\vf{k} 3}$ be eigenvectors of $\Phi(\vf{k})$, which correspond to eigenvalues $\Phi_1$, $\Phi_2$, and $\Phi_3$
% \item Eq.~(\ref{eq:phonon_motion_matrix}) gives
% \begin{align}
% \omega_{\vf{k} \nu}^2 = \frac{\Phi_\nu(\vf{k})}{M},~\nu = 1, 2, 3.
% \end{align}
% \item The three polarization vectors $\vf{A}_{\vf{k} \nu}$ comprise one longitudinal mode where $\vf{A}$ points along $\vf{k}$, and two transverse modes where $\vf{A}$ is perpendicular to $\vf{k}$
% \end{itemize}
% \end{frame}

% \begin{frame}{$\Phi^{l l^{\prime}}$ must obey some important symmetries}
% \begin{itemize}
% \item The energy of the crystal cannot change if all ions are simultaneously displaced by a single vector. Thus,
% \begin{align}
% \sum_{l^{\prime}} \Phi^{l l^{\prime}} = 0_{3,3}. \mick{\text{why?}}
% \end{align}
% From Eq.~(\ref{eq:dyn_matrix}), we have
% \begin{align}
% \Phi(\vf{k} = \vf{0}) = 0_{3,3}.
% \end{align}
% \item Applying periodic boundary conditions to Eq.~(\ref{eq:phonon_motion_matrix}) gives
% \begin{align}
% \Phi(\vf{k} + \vf{K}) = \Phi(\vf{k}),
% \end{align}
% where $\vf{K}$ is any reciprocal lattice vector.
% \end{itemize}
% \end{frame}

% \begin{frame}{Lattice with a basis leads to more than one branch}
% \begin{itemize}
% \item The same phenomenon persists in 3D
% \item Adding new atoms to a unit cell adds new degrees of freedom to the lattice
% \item One must consider a correspondingly greater number of normal modes to describe them
% \item For example, in three dimensions with four atoms per unit cell, one has $3 \times 4$ normal modes for every value of $\vf{k}$
% \item The equation of motion is now
% \begin{align}
% M_n \ddot{\vf{u}}^{l n} = -\sum_{l^{\prime} n^{\prime}} \Phi^{l n l^{\prime} n^{\prime}} \vf{u}^{l^{\prime} n^{\prime}},
% \end{align}
% where the superscripts $n$ and $n^{\prime}$ label the different atoms comprising the basis in each unit cell.
% \end{itemize}
% \end{frame}

% \begin{frame}
% \begin{itemize}
% \item Following the same procedure, the solution is taken to be
% \begin{align}
% \vf{u}^{l n} = \vf{A}^n e^{i \vf{k} \cdot \vf{R}^{l n} - i \omega t},
% \end{align}
% which leads to
% \begin{align}
% M_n \omega^2 \vf{A}^n = \sum_{n^{\prime}} \Phi^{n n^{\prime}}(\vf{k}) \vf{A}^{n^{\prime}}.
% \end{align}
% \item To simplify the problem, we define a new index $p$ that ranges over all degrees of freedom in the unit cell
% \item With four atoms per unit cell, $p$ would range from 1 to 12
% \item Using this notation,
% \begin{align}
% M_p \omega^2 A_p = \sum_{p^{\prime}}^{3N} \Phi_{p p^{\prime}} (\vf{k}) A_{p^{\prime}}.
% \end{align}
% \end{itemize}
% \end{frame}

% \begin{frame}{Example: diamond lattice}
% \begin{itemize}
% \item Suppose one has a collection of atoms sitting on a Bravais lattice $\vf{R}^{l}$ with basis $\vf{v}^n$
% \item Take $\vf{R}^{l n} = \vf{R}^l + v^n$, and have particles interact with a potential of the form
% \begin{align}
% U = \frac{1}{2} \sum_{l n l^{\prime} n^{\prime}} \phi_{n n^{\prime}} (|\vf{u}^{ln} + \vf{R}^{ln} - \vf{u}^{l^{\prime} n^{\prime}} - \vf{R}^{l^{\prime} n^{\prime}}|) \mick{\text{What sum exactly?}}.
% \end{align}
% Expanding to quadratic order in the small deviations $\vf{u}$ and using the fact that terms linear in $\vf{u}$ must vanish, we find
% \begin{align}
% U \approx \frac{1}{4} \sum_{l n l^{\prime} n^{\prime}} (\vf{})
% \end{align}
% \end{itemize}
% \end{frame}

\end{document}
